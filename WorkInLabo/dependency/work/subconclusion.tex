\subsection*{Sub Conclusion}

\paragraph{For Artificial Data}
\begin{itemize}
	\item There are 3 types of functions for representing theta: 'quadra', 'sharp peak', 'triangle': Figure \ref{bandit-functions0.4}.
		\begin{figure}[h]
		\centering
		\includegraphics[scale=0.5]{resources/0717/bandit-functions.png}
		\caption{Bandit functions XOPT=0.4}
		\label{bandit-functions0.4}
		\end{figure}
		
	\item When algo samples arm $x$, we use Normal distribution to add noise in the reward return to algo: Norm(mean: $theta(x)$, standard deviation: $0.2 + theta(x)/1.2$) where $theta(x)$ is calculated by function in Figure \ref{bandit-functions0.4}.
	
	\item For LSE, LSE-backtrack, LSE-weighted, we should define number of sampling(NS). We have executed algorithms by setting NS from 8 to 20.
	
	\item To avoid contingency, we have use rounds from 10 to 80(ex. execute 80 times for each algorithm and calculate the mean value).
	
	\item For function 'Sharp Peak', $X_opt$ = 0.4, T = 15000, rounds = 80, number of sampling = 10, the curves of approximation and regret:Figure \ref{approx-sharp} \ref{regret-sharp}
	
	\begin{figure}[h]
		\centering
		\includegraphics[scale=0.5]{resources/0717/approx-sharp.png}
		\caption{Approximation XOPT = 0.4}
		\label{approx-sharp}
		\end{figure}
		
	\begin{figure}[h]
		\centering
		\includegraphics[scale=0.5]{resources/0717/regret-sharp.png}
		\caption{Regret XOPT=0.4}
		\label{regret-sharp}
		\end{figure}
	
	\item Intervals. To test the performance of LSE-backtrack, it is necessary to set the interval initial $(x_i, x_j)$ where $x_i < x_j \wedge x_{opt} \notin [x_i, x_j]$ 
	
	\begin{figure}[h]
		\centering
		\includegraphics[scale=0.5]{resources/0717/approx-sharp.png}
		\caption{Approximation XOPT = 0.4}
		\label{approx-sharp}
		\end{figure}
		
	\begin{figure}[h]
		\centering
		\includegraphics[scale=0.5]{resources/0717/regret-sharp.png}
		\caption{Regret XOPT=0.4}
		\label{regret-sharp}
		\end{figure}


\end{itemize}



